\documentclass[12pt]{article}
\NeedsTeXFormat{LaTeX2e}

%%%%%%%%%%%%%%%%%%%%%%%%%%%%%%%%%%%%%%%%%

% Template by Nicholas Bertollo
% You may change and reuse this as you see fit

%%%%%%%%%%%%%%%%%%%%%%%%%%%%%%%%%%%%%%%%%

% Replace this information!

\newcommand{\sid}{123456789} 
\newcommand{\compunit}{COMP2X23}
\newcommand{\ass}{X}

%%%%%%%%%%%%%%%%%%%%%%%%%%%%%%%%%%%%%%%%%

\usepackage[svgnames]{xcolor}
\usepackage[T1]{fontenc}
\usepackage[margin=2.7cm,a4paper]{geometry}

\usepackage[osf]{mathpazo}
\usepackage{amsmath,amsthm,amsfonts,amssymb,mathtools}

\usepackage{hyperref,url}
\usepackage{cleveref}
\usepackage{xstring}

\usepackage{environ}
\usepackage{tasks}

\usepackage{etoolbox}
\usepackage{fourier-orns}
\usepackage{kvoptions}
\usepackage[]{units}
\usepackage[normal]{subfigure}

% Algorithms package

% \usepackage[ruled]{algorithm2e} % You can use the algorithm2e if you'd like
\usepackage[noend]{algpseudocode} % You may get rid of noend if you like. 
\usepackage{algorithm}
\usepackage{algorithmicx}

% Header

\usepackage{fancyhdr}
\addtolength{\headheight}{2.5pt}
\pagestyle{fancy}
\fancyhead{} 
\fancyhead[L]{\sc \compunit}
\renewcommand{\headrulewidth}{0.75pt}
\fancyhead[C]{\sc SID: \sid}
\fancyhead[R]{Assignment \ass}

%%%%%%%%%%%%%%%%%%%%%%%%%%%%%%%%%%%%%%%%%%%%%%%%%%%%%%%%%%%%%%%%%%%%%%%%%%%%%%%%%%

% Some basic mathematics commands!

\DeclarePairedDelimiter\ceil{\lceil}{\rceil}
\DeclarePairedDelimiter\floor{\lfloor}{\rfloor}
\DeclarePairedDelimiter\abs{\lvert}{\rvert}

\newcommand{\Z}{\mathbb{Z}}
\newcommand{\N}{\mathbb{N}}
\newcommand{\Q}{\mathbb{Q}}
\newcommand{\R}{\mathbb{R}}
\newcommand{\C}{\mathbb{C}}

\newcommand{\map}[2]{\,{:}\,#1\!\longrightarrow\!#2}

% fancy thing I found on the internet

\newcommand*\Let[2]{\State #1 $\gets$ #2}

% The Solution!

\newcommand{\solution}[1]{\vspace{.2cm}\noindent \textbf{Solution #1)}}

%%%%%%%%%%%%%%%%%%%%%%%%%%%%%%%%%%%%%%%%%%%%%%%%%%%%%%%%%%%%%%%%%%%%%%%%%%%%%%%%%%
 
\begin{document}

\solution{1} 
This is just a couple examples of what you can do in \LaTeX. 
You may delete it all. 

\solution{1a} 
It isn't considered sufficient to explain your algorithm through pseudo-code only, however here is a basic example using the \href{https://ctan.org/pkg/algorithmicx?lang=en}{\textbf{algorithmicx package}}.

\begin{algorithm}
    \caption{This squares every element of a list} \label{alg:pre}
    \begin{algorithmic}[1] % [1] adds line numbers!
        \Require Integer list of size $n$
        \Statex
        \Function{square}{\textit{elements}}
            \For{(index, element) $\in \{ 1, 2, \cdots, n \} \times \textit{elements}$}
                \Let{list[index]}{element $\cdot$ element}
            \EndFor
        \EndFunction
    \end{algorithmic}
\end{algorithm}

\solution{1b}
\Cref{alg:pre} is $O(n)$ because if $T(n)$ is a function which counts the number of $O(1)$ operations, then $T(n)$ is upper bounded by some constant multiple of $n$, i.e. there exists $C$ such that $T(n) \le C \cdot n$ for all $n$ large enough. 

\solution{1c} 
For a predicate $P$, $P(0) \wedge (P(k) \to P(k + 1)) \to (\forall k \in \N, P(k))$. 
A function $x\map{\N}{\R}$ can be represented as a sequence $\{x_n\}_{n \in \N}$.
$\log_2(n) = \log(n)$ in COMP2123!

\newpage
\solution{2a}
Start a new question on a new page each time to make it easier and to hand in on gradescope.

\end{document}